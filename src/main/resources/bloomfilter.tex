\title{Bloom-Filter}
\author{
         Severin Peyer, Daniel Obrist\\
}
\date{\today}

\documentclass[12pt]{article}

\begin{document}
\maketitle

\section{Idee des Bloom-Filters}
Der Bloom-Filter ist eine pobabilistische Datenstruktur um schnell und speichereffizient zu bestimmen, ob ein Element in einem Set vorhanden ist. Er kann uns sagen, ob ein Element entweder definitiv nicht im Set ist oder sich mit einer gewissen Fehlerwahrscheinlichkeit im Set befinden kann.

Ein Bloom-Filter ist ein Array von \textit{m} Bits, die anfangs alle auf 0/false gesetzt sind. Wird ein Element in das Set hinzugefügt, werden durch \textit{k} unterschiedliche Hash-Funktionen Hashcodes berechnet. Diese Hashcodes bestimmmen, welche Stellen des Bloomfilters auf 1/true gesetzt werden. So entsteht eine Art Fingerabdruck aller hinzugefügten Daten. Beim Überprüfen ob ein Element enhalten ist werden ebenfalls die Hashcodes des Ekements berechnet und mit dem Bloom-Filter verglichen. Wenn an einer der Positionen 0/false steht, ist das Element definitiv nicht im Set. Es können aber False-Positives entstehen, wenn per Zufall zwei Elemente die gleichen Hashcodes generieren.  Je mehr Elemente zur Menge hinzugefügt wurden, desto grösser wird diese Fehlerwahrscheinlichkeit \textit{p}, dass man aus Versehen eine positive Antwort bekommt, obwohl das Element gar nicht im Set enthalten ist.


\paragraph{Vorteile}
\begin{itemize}
\item Zeit, um  Elemente hinzuzufügen oder zu überprüfen, ob ein Element im Set ist konstant, O(k)
\item Benötigen den eigentlichen Speicher/die Daten gar nicht
\item Fehlerwahrscheinlichkeit kann festgelegt werden und durch mehrere Hash-Funktionen oder die Filtergrösse gesenkt werden
\end{itemize}

\paragraph{Nachteile}
\begin{itemize}
\item Eigentliche Daten müssen separat abgespeichert werden
\item Elemente entfernen ist mühsam
\item Filter kann sehr gross werden und muss ausbalanciert werden
\end{itemize}

\section{Praxisbeispiel}
Beispiele sind NoSQL-Datenbanken wie BigTable, Apache HBase und Apache Cassandra. Diese Datenbanken verwenden Bloom-Filter für das Nachschlagen eines Schlüssels in einer grösseren Tabelle. Der Bloom-Filter wird dabei der eigentlichen Suche vorgeschaltet um unnötige Suchvorgänge einzusparen. Nur wenn der Wert gemäss Bloom-Filter in der Tablle enthalten sein könnte, wird die aufwändige Suche in den Datenwerten angestossen.

\section{Testen der Fehlerwahrscheinlichkeit}
In this section we describe the results.

\end{document}
This is never printed